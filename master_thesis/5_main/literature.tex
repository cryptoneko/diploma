\section{Попередні роботи}

У даній роботі використовується породжувальна модель обличчя
3D Basel Face Model (BFM)
розроблена командою Базельського університету
\cite{bfm09}.
Перший підхід,
яких дозволяє отримати тривимірну модель обличчя лише за одним фото,
було запропоновано Томасом Феттером та Фолькером Бланцом
\cite{blanz:vetter:1999}.
Вони використовували BFM та стохастичний градієнтний спуск \cite{sgd:1998},
що дозволяло досягти добрих результатів за 40 хвилин з процесором
Pentium III, 800MHz \cite{blanz:romdhani:vetter}.
Згодом ті ж автори розробили та використали у своїй роботі
стохастичний алгоритм Ньютона \cite{blanz:vetter:2003},
яким досягли реконструкції просторової конфігурації обличчя за 4.5 хвилини на
Pentium 4, 2GHz.

Пізніше студент Національного Технічного Университету України
``Київський Політехнічний Інститут'' запропонував та запатентував
свій метод створення генеративної моделі обличчя та відновлення
тривимірної поверхні обличчя за одним або кількома фото \cite{tyshchenko2011}.
Основні відмінності полягають у тому,
що задачу співставлення відповідних точок різних моделей
було представлено та розв'язано в термінах задачі розмітки
\cite{Rossi:2006:HCP:1207782},
а нові моделі облич генеруються як зважене середнє.
